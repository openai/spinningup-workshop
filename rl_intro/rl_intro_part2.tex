\documentclass[9pt]{beamer}
\usepackage{beamerthemesplit}
%\usepackage{times}
\usepackage{graphicx}
\usepackage{algorithm}
\usepackage{algorithmic}
\usepackage{amsmath}
\usepackage{amssymb}
\usepackage{amsthm}
%\usepackage{thmtools,thm-restate}
\usepackage[font=small]{caption}
\usepackage{subcaption}
\usepackage{listings}
\usepackage{xcolor}
\usepackage{float}
\usepackage{animate}
\usepackage{media9}
\usepackage{multimedia}
\usepackage{hyperref}
\usepackage{cancel}

\usetheme{Madrid}

\graphicspath{{./images/}}

\AtBeginSection[]{
  \begin{frame}
  \vfill
  \centering
  \begin{beamercolorbox}[sep=8pt,center,shadow=true,rounded=true]{title}
    \usebeamerfont{title}\insertsectionhead\par%
  \end{beamercolorbox}
  \vfill
  \end{frame}
}


\newcommand{\underE}[2]{\underset{\begin{subarray}{c}#1 \end{subarray}}{\E}\left[ #2 \right]}


\newcommand\Fontvi{\fontsize{6}{7.2}\selectfont}
\renewcommand{\d}[1]{\ensuremath{\operatorname{d}\!{#1}}}

\newcommand{\twocolumns}[4]{
\begin{columns}
\begin{column}{#1\textwidth}
    #3
\end{column}
\begin{column}{#2\textwidth}
	#4
\end{column}
\end{columns}
}

\usepackage{listings}
\usepackage{color}
\usepackage{courier}
 
\definecolor{codegreen}{rgb}{0,0.6,0}
\definecolor{codegray}{rgb}{0.5,0.5,0.5}
\definecolor{codepurple}{rgb}{0.58,0,0.82}
\definecolor{backcolour}{rgb}{0.95,0.95,0.92}


\lstdefinestyle{mystyle}{
    backgroundcolor=\color{backcolour},   
    commentstyle=\color{codegreen},
    keywordstyle=\color{magenta},
    numberstyle=\tiny\color{codegray},
    stringstyle=\color{codepurple},
    basicstyle=\footnotesize\ttfamily,
    breakatwhitespace=false,         
    breaklines=true,                 
    captionpos=b,                    
    keepspaces=true,                 
    %numbers=left,                    
    numbersep=5pt,                  
    showspaces=false,                
    showstringspaces=false,
    showtabs=false,                  
    tabsize=2
}
 
\lstdefinestyle{mystyle2}{
    backgroundcolor=\color{backcolour},   
    commentstyle=\color{codegreen},
    keywordstyle=\color{magenta},
    numberstyle=\tiny\color{codegray},
    stringstyle=\color{codepurple},
    basicstyle=\tiny\ttfamily,
    breakatwhitespace=false,         
    breaklines=true,                 
    captionpos=b,                    
    keepspaces=true,                 
    %numbers=left,                    
    numbersep=5pt,                  
    showspaces=false,                
    showstringspaces=false,
    showtabs=false,                  
    tabsize=2
}

\lstdefinestyle{mystyle3}{
    backgroundcolor=\color{backcolour},   
    commentstyle=\color{codegreen},
    keywordstyle=\color{magenta},
    numberstyle=\tiny\color{codegray},
    stringstyle=\color{codepurple},
    basicstyle=\scriptsize\ttfamily,
    breakatwhitespace=false,         
    breaklines=true,                 
    captionpos=b,                    
    keepspaces=true,                 
    %numbers=left,                    
    numbersep=5pt,                  
    showspaces=false,                
    showstringspaces=false,
    showtabs=false,                  
    tabsize=2
}

\lstset{style=mystyle3}


\begin{document}
%
% Definitions and macros
%

%\setlength{\marginparwidth}{1.2in}
%\let\oldmarginpar\marginpar
%\renewcommand\marginpar[1]{\-\oldmarginpar[\raggedleft\footnotesize #1]%
%{\raggedright\footnotesize #1}}

%\renewcommand{\indexspace}{\rule{0cm}{.4cm}}
%% end example/remark
%\newcommand{\eex}{\ifmmode\sq\else{\unskip\nobreak\hfil
%  \penalty50\hskip1em\null\nobreak\hfil$\Diamond$
%  \parfillskip=0pt\finalhyphendemerits=0\endgraf}\fi{}}
%\newcommand{\erem}{\ifmmode\sq\else{\unskip\nobreak\hfil
%  \penalty50\hskip1em\null\nobreak\hfil$\star$
%  \parfillskip=0pt\finalhyphendemerits=0\endgraf}\fi{}}
%\newcommand{\eobs}{\ifmmode\sq\else{\unskip\nobreak\hfil
%  \penalty50\hskip1em\null\nobreak\hfil$\vartriangleleft$
%  \parfillskip=0pt\finalhyphendemerits=0\endgraf}\fi{}}

% VET - characters - lowercase
\newcommand{\avet}{{\mathbf  a}}
\newcommand{\bvet}{{\mathbf  b}}
\newcommand{\cvet}{{\mathbf  c}}
\newcommand{\dvet}{{\mathbf  d}}
\newcommand{\evet}{{\mathbf  e}}
\newcommand{\fvet}{{\mathbf  f}}
\newcommand{\gvet}{{\mathbf  g}}
\newcommand{\hvet}{{\mathbf  h}}
\newcommand{\ivet}{{\mathbf  i}}
\newcommand{\jvet}{{\mathbf  j}}
\newcommand{\kvet}{{\mathbf  k}}
\newcommand{\lvet}{{\mathbf  l}}
\newcommand{\mvet}{{\mathbf  m}}
\newcommand{\nvet}{{\mathbf  n}}
\newcommand{\ovet}{{\mathbf  o}}
\newcommand{\pvet}{{\mathbf  p}}
\newcommand{\qvet}{{\mathbf  q}}
\newcommand{\rvet}{{\mathbf  r}}
\newcommand{\svet}{{\mathbf  s}}
\newcommand{\tvet}{{\mathbf  t}}
\newcommand{\uvet}{{\mathbf  u}}
\newcommand{\vvet}{{\mathbf  v}}
\newcommand{\xvet}{{\mathbf  x}}
\newcommand{\yvet}{{\mathbf  y}}
\newcommand{\zvet}{{\mathbf  z}}
\newcommand{\wvet}{{\mathbf  w}}

% VET - characters - uppercase
\newcommand{\Avet}{{\mathbf  A}}
\newcommand{\Bvet}{{\mathbf  B}}
\newcommand{\Cvet}{{\mathbf  C}}
\newcommand{\Dvet}{{\mathbf  D}}
\newcommand{\Evet}{{\mathbf  E}}
\newcommand{\Fvet}{{\mathbf  F}}
\newcommand{\Gvet}{{\mathbf  G}}
\newcommand{\Hvet}{{\mathbf  H}}
\newcommand{\Ivet}{{\mathbf  I}}
\newcommand{\Jvet}{{\mathbf  J}}
\newcommand{\Kvet}{{\mathbf  K}}
\newcommand{\Lvet}{{\mathbf  L}}
\newcommand{\Mvet}{{\mathbf  M}}
\newcommand{\Nvet}{{\mathbf  N}}
\newcommand{\Ovet}{{\mathbf  O}}
\newcommand{\Pvet}{{\mathbf  P}}
\newcommand{\Qvet}{{\mathbf  Q}}
\newcommand{\Rvet}{{\mathbf  R}}
\newcommand{\Svet}{{\mathbf  S}}
\newcommand{\Tvet}{{\mathbf  T}}
\newcommand{\Uvet}{{\mathbf  U}}
\newcommand{\Xvet}{{\mathbf  X}}
\newcommand{\Yvet}{{\mathbf  Y}}
\newcommand{\Vvet}{{\mathbf  V}}
\newcommand{\Wvet}{{\mathbf  W}}
\newcommand{\Zvet}{{\mathbf  Z}}

\newcommand{\Deltavet}{\mathbf  \Delta}
\newcommand{\Lambdavet}{{\mathbf  \Lambda}}
\newcommand{\Sigmavet}{\mathbf  \Sigma}
\newcommand{\Thetavet}{{\mathbf  \Theta}}

% Special characters:
\newcommand{\s}{ {\sigma} }

\newcommand{\e}{{\mathrm e}}
\newcommand{\jm}{{\mathrm j}}
\newcommand{\E}{{\mathrm E}}
\newcommand{\Ex}{{\mathbb E}}
\renewcommand{\d}{{\mathrm d}}
\newcommand{\dt}{{\mathrm d}t}
\newcommand{\X}{ {\mathcal X} }
\newcommand{\Y}{ {\mathcal Y} }
\newcommand{\Z}{ {\mathcal Z} }



\newcommand{\calA}{{\mathcal A}}
\newcommand{\calB}{{\mathcal B}}
\newcommand{\calC}{{\mathcal C}}
\newcommand{\calD}{{\mathcal D}}
\newcommand{\calE}{{\mathcal E}}
\newcommand{\calF}{{\mathcal F}}
\newcommand{\calG}{{\mathcal G}}
\newcommand{\calH}{{\mathcal H}}
\newcommand{\calI}{{\mathcal I}}
\newcommand{\calJ}{{\mathcal J}}
\newcommand{\calK}{{\mathcal K}}
\newcommand{\calL}{{\mathcal L}}
\newcommand{\calM}{{\mathcal M}}
\newcommand{\calN}{{\mathcal N}}
\newcommand{\calO}{{\mathcal O}}
\newcommand{\calP}{{\mathcal P}}
\newcommand{\calQ}{{\mathcal Q}}
\newcommand{\calR}{{\mathcal R}}
\newcommand{\calS}{{\mathcal S}}
\newcommand{\calT}{{\mathcal T}}
\newcommand{\calU}{{\mathcal U}}
\newcommand{\calV}{{\mathcal V}}
\newcommand{\calX}{{\mathcal X}}
\newcommand{\calY}{{\mathcal Y}}
\newcommand{\calW}{{\mathcal W}}
\newcommand{\calZ}{{\mathcal Z}}
\newcommand{\qtil}{{\tilde{q}}}
\newcommand{\td}{{\tilde{\delta}}}

\newcommand{\vect}[1]{ {\mbox{\rm vec}(#1)} }

% Macro comandi:

\newcommand{\Atil}{\tilde{A}}
\newcommand{\Zhat}{\hat{Z}}
\newcommand{\Hbar}{\bar{H}}
\newcommand{\Dhat}{\hat{D}}
\newcommand{\dhat}{\hat{d}}
%

\newcommand{\rhat}{\hat{r}}
\newcommand{\xhat}{\hat{x}}
\newcommand{\yhat}{\hat{y}}
\newcommand{\zhat}{\hat{z}}
\newcommand{\xbar}{\bar{x}}
\newcommand{\ubar}{\bar{u}}
\newcommand{\ybar}{\bar{y}}
\newcommand{\zbar}{\bar{z}}
%
\newcommand{\pdot}{\dot{p}}
\newcommand{\pddot}{\ddot{p}}
\newcommand{\pbar}{\bar{p}}
%
\newcommand{\qdot}{\dot{q}}
\newcommand{\qddot}{\ddot{q}}
\newcommand{\qbar}{\bar{q}}
%
\newcommand{\xdot}{\dot{x}}
\newcommand{\ydot}{\dot{y}}
\newcommand{\zdot}{\dot{z}}
\newcommand{\yddot}{\ddot{y}}
\newcommand{\thdot}{\dot{\theta}}
\newcommand{\thddot}{\ddot{\theta}}
\newcommand{\util}{{\tilde{u}}}
\newcommand{\xtil}{{\tilde{x}}}
\newcommand{\ytil}{{\tilde{y}}}
\newcommand{\lam}{\lambda}
\newcommand{\lamax}{\lambda\ped{max}}
\newcommand{\lamin}{\lambda\ped{min}}
%
\newcommand{\adj}{ {\mbox{\rm adj}\;} }
\newcommand{\sign}{\mbox {\rm sgn}}
\newcommand{\spn}{\mbox {\rm span}}
\newcommand{\barJ}{\bar{J}}
\newcommand{\dom}{\mathop {\mathrm {dom}}}
\newcommand{\card}{\mathop{\mathrm{card}}}
\newcommand{\subt}{\mathop{\mathrm{s.t.}}}

\newcommand{\epi}{\mathop{\mathrm{epi}}}
\newcommand{\env}{\mathop{\mathrm{env}}}
\newcommand{\chull}{\mathop{\mathrm{co}}}
\newcommand{\graph}{\mathop{\mathrm{graph}}}
\newcommand{\prox}[1]{\mathop{\mathrm{prox}_{#1}}}
\newcommand{\sthr}[1]{\mathop{\mathrm{sthr}_{#1}}}

\def\hardsection{$\spadesuit\;$}





%%%% Fields and Groups
\newcommand{\Real}[1]{ { {\mathbb R}^{#1} } }
\newcommand{\Realp}[1]{ { {\mathbb R}_{+}^{#1} } }
\newcommand{\Realpp}[1]{ { {\mathbb R}_{++}^{#1} } }
\newcommand{\Complex}[1]{ { {\mathbb C}^{#1} } }
\newcommand{\Imag}[1]{ { {\mathbb I}^{#1} } }
\newcommand{\Field}[1]{ {\mathbb F}^{#1} }
\newcommand{\F}{ {\mathbb F}}
\newcommand{\Orth}[1]{ { {\calG_{\calO}^{#1}} } }
\newcommand{\Unit}[1]{ { {\calG_{\calU}^{#1}} } }
\newcommand{\Sym}[1]{ { {\mathbb S}^{#1} } }
\newcommand{\Symp}[1]{ { {\mathbb S}_{+}^{#1} } }
\newcommand{\Sympp}[1]{ { {\mathbb S}_{++}^{#1} } }
\newcommand{\Herm}[1]{ { {\mathbb H}^{#1} } }
\newcommand{\Skew}[1]{ { {\mathbb S\mathbb K}^{#1} } }
\newcommand{\Skherm}[1]{ { {\mathbb H\mathbb K}^{#1} } }
% manifolds (in matrices)
\newcommand{\Rman}[1]{ { {\mathcal R}^{#1} } } 
\newcommand{\Cman}[1]{ { {\mathcal C}^{#1} } }
%
\newcommand{\Hinf}[1]{ {  {\mathcal H}_\infty^{#1} } }
\newcommand{\RHinf}[1]{ { {\mathcal RH}_\infty^{#1} } }
\newcommand{\Htwo}[1]{ {  {\mathcal H}_2^{#1} } }
\newcommand{\RHtwo}[1]{ { {\mathcal RH}_2^{#1} } }

\newcommand{\dist}[1]{{\mathrm{dist}}{\left( #1 \right)}}
%
\newcommand{\diff}[2]{ \frac{\d {#1}}{\d {#2}}  }
\newcommand{\diffp}[2]{ \frac{\partial {#1}}{\partial {#2}}  }
\newcommand{\diffqd}[2]{ \frac{\d^2 {#1}}{\d {#2}^2}  }
\newcommand{\diffq}[2]{ \frac{\d^2 {#1}}{\d {#2}}  }
\newcommand{\diffqq}[3]{ \frac{\d^2 {#1}}{ \d {#2} \d {#3}  }}
\newcommand{\diffpq}[2]{ \frac{\partial^2 {#1}}{\partial {#2}^2}  }
\newcommand{\difftq}[3]{ \frac{\partial^2 {#1}}{\partial {#2}\partial {#3}}  }
\newcommand{\diffi}[3]{ \frac{\d^{#3} {#1}}{\d {#2}^{#3}}  }
\newcommand{\diffpi}[3]{ \frac{\partial^{#3} {#1}}{\partial {#2}^{#3}}  }
\newcommand{\binomial}[2]{\scriptsize{\left(\!\! \ba{c} #1 \\ #2 \ea \!\! \right)} }
\newcommand{\comb}[2]{{\left(\!\!\! \ba{c} #1 \\ #2 \ea \!\!\! \right)} }

\newcommand{\simax}{{\sigma_{\mathrm{max}}}}
\newcommand{\simin}{{\sigma_{\mathrm{min}}}}
\newcommand{\prob}{{\mbox{\rm Prob}}}
\newcommand{\var}{{\mbox{\rm var}}}
\newcommand{\sint}{{\mbox{\rm int}\,}} %set interior
\newcommand{\relint}{{\mbox{\rm relint}\,}} %set interior
\newcommand{\ns}{{\mbox{\tt ns}}} 

%

\newcommand{\rank}{\mathop{\mathrm{rank}}\nolimits}
\newcommand{\range}{\mathop{\mathcal{R}}\nolimits}
\newcommand{\nulsp}{\mathop{\mathcal{N}}\nolimits}
\newcommand{\diagop}{\mathop{\mathrm{diag}}\nolimits}
\newcommand{\Var}{\mathop{\mathrm{var}}\nolimits}
\newcommand{\tr}{\mathop{\mathrm{trace}}\nolimits}
\newcommand{\sinc}{\mathop{\mathrm{sinc}}\nolimits}

%%%% Real and Imaginary
\newcommand{\pre}[1]{ { {\mathop{\mathrm{Re}}}  \left({#1}\right)} }
\newcommand{\pim}[1]{ { {\mathop{\mathrm{Im}}}  ({#1})} }
\newcommand{\rp}{ ^{\Real{}} }
\newcommand{\ip}{ ^{\Imag{}} }



%%%% Various
\newcommand{\one}{{\mathbf  1}}
%\newcommand{\qed}{{\hfill $\square$}}
\newcommand{\dss}{\displaystyle}
\newcommand{\inv}{^{-1}}
\newcommand{\pinv}{^{\dagger}}
\newcommand{\diag}[1]{\mathrm{diag}\left({#1}\right)}
\newcommand{\blockdiag}[1]{\mbox{\rm bdiag}\left({#1}\right)}
\newcommand{\tran}{^{\top}}
\newcommand{\inner}[1]{\langle {#1} \rangle}
\newcommand{\ped}[1]{_{\mathrm{#1}}}
\newcommand{\ap}[1]{^{\mathrm{#1}}}

\newcommand{\blu}[1]{\textcolor{blue}{#1}}
\newcommand{\red}[1]{\textcolor{red}{#1}}
\newcommand{\green}[1]{\textcolor{green}{#1}}
\newcommand{\cyan}[1]{\textcolor{cyan}{#1}}
\newcommand{\comment}[1]{\vspace{.1cm} \blu{#1} \vspace{.1cm}}



%%%% Commands
\newcommand{\beq}{\begin{equation}}
\newcommand{\eeq}{\end{equation}}
\newcommand{\bea}{\begin{eqnarray}}
\newcommand{\eea}{\end{eqnarray}}
\newcommand{\beas}{\begin{eqnarray*}}
\newcommand{\eeas}{\end{eqnarray*}}
\newcommand{\ba}{\begin{array}}
\newcommand{\ea}{\end{array}}
\newcommand{\bit}{\begin{itemize}}
\newcommand{\eit}{\end{itemize}}
\newcommand{\ben}{\begin{enumerate}}
\newcommand{\een}{\end{enumerate}}
\newcommand{\bde}{\begin{description}}
\newcommand{\ede}{\end{description}}
\newcommand{\bsp}{\begin{split}}
\newcommand{\esp}{\end{split}}

%% Environments
%\newtheorem{corollary}{Corollary}
%\newtheorem{exercise}{Exercise}
%\newtheorem{solution}{Solution}
%\newtheorem{algorithm}{Algorithm}
%\newtheorem{assumption}{Assumption}
%\newtheorem{definition}{Definition}
%\newtheorem{proposition}{Proposition}
%\newtheorem{procedure}{Procedure}
%\newtheorem{lemma}{Lemma}
%\newtheorem{fact}{Fact}
%\newtheorem{example}{Example}
%\theoremstyle{remark}
%\newtheorem{theorem}{Theorem}
%\newtheorem*{remark}{Remark}

%%% margin stuff
%% example in the margin
%\newcommand{\marginex}[1]{
%\marginnote{\refstepcounter{examplectr}{\bfseries\textsf{Example \theexamplectr.}} 
%#1
%}}
%
%% remark in the margin
%\newcommand{\marginrmk}[1]{
%\marginnote{\refstepcounter{remarkctr}{\bfseries\textsf{Remark \theremarkctr.}} 
%#1
%}}
%
%% algorithm in the margin
%\newcommand{\marginalg}[1]{
%\marginnote{\refstepcounter{algorithmctr}{\bfseries\textsf{Algorithm \thealgorithmctr.}} 
%#1
%}}

%% misc
% these commands are to make things compile
\def\nocolon{}

% Prints the month name (e.g., January) and the year (e.g., 2008)
\newcommand{\monthyear}{%
  \ifcase\month\or January\or February\or March\or April\or May\or June\or
  July\or August\or September\or October\or November\or
  December\fi\space\number\year
}

% Prints an epigraph and speaker in sans serif, all-caps type.
\newcommand{\openepigraph}[2]{%
  %\sffamily\fontsize{14}{16}\selectfont
  \begin{fullwidth}
  \sffamily\large
  \begin{doublespace}
  \noindent\allcaps{#1}\\% epigraph
  \noindent\allcaps{#2}% author
  \end{doublespace}
  \end{fullwidth}
}

% Inserts a blank page
\newcommand{\blankpage}{\newpage\hbox{}\thispagestyle{empty}\newpage}




\title{Introduction to Deep RL, Part 2}
\author{Joshua Achiam}
\institute{OpenAI}
\titlegraphic{\includegraphics[height=2cm]{spinning-up-logo2}}
%\date{March 3, 2018}

\begin{frame}
\titlepage
\end{frame}


\section{2.1: What is RL currently achieving?}

\begin{frame}{What RL Can Currently Do: RL in Simulation}

\twocolumns{0.5}{0.5}{
If you have infinite simulator data and well-defined rewards, you can make substantial progress on extremely hard problems!
\begin{itemize}
\item Atari
\item Simulated robotics
\item Go (Deepmind's AlphaZero)
\item Dota (OpenAI Five)
\item Starcraft (Deepmind's AlphaStar)
\end{itemize}
}{
\includegraphics[width=\textwidth]{p2-five}

\includegraphics[width=\textwidth]{p2-alphastar}
}
\end{frame}

\begin{frame}{Spotlight on AlphaGo}

\twocolumns{0.6}{0.4}{
\includegraphics[width=\textwidth]{p2-alphago}
}{
Hard to overstate what an unbelievable accomplishment this was

\vspace{2em}

Previously: Go was considered unassailable stronghold for human experts, AI 10+ years away
}

\end{frame}

\begin{frame}{What RL Can Currently Do: RL in the Real World}

\twocolumns{0.5}{0.5}{
RL is beginning to see profitable real-world applications!
\begin{itemize}
\item Facebook uses RL (DQN) for push notifications (Horizon)
\item DeepMind integrated RL into data center cooling 
\item Several promising early efforts at applying RL for robotics
\end{itemize}
}{

\includegraphics[width=\textwidth]{p2-horizon}


\includegraphics[width=\textwidth]{p2-datacenter}
}

\end{frame}

\section{2.1.1: Spotlight on RL for Real-World Robotics}


\begin{frame}{Tasks that can't easily be simulated}

Zhu et al. trained low-cost robots from scratch in the real world on hard-to-simulate tasks using natural policy gradient. (Note: observation space here is hand state and valve state)

\begin{center}
\movie[width=0.8\textwidth, autostart, loop]{\includegraphics[width=0.8\textwidth]{p2-foamscrew}}{images/p2-foamscrew.mp4}
\end{center}


\footnotemark[1]{Zhu et al, 2018: ``Dexterous Manipulation with Deep Reinforcement Learning: Efficient, General, and Low-Cost"}

\end{frame}

\begin{frame}{Robots that are hard to control conventionally}

Zhang et al. trained a tensegrity robot with deep RL in simulation and demonstrated transfer to the real world
\begin{figure}
\centering
\includegraphics[width=0.8\textwidth]{p2-tensegrity}
\end{figure}


\footnotemark[1]{Zhang et al, 2016: ``Deep Reinforcement Learning for Tensegrity Robot Locomotion"}
\end{frame}

\begin{frame}{Soft Actor-Critic}

Haarnoja et al. trained robots in the real world with Soft Actor-Critic, and demonstrated efficient and robust control on hard domains

\twocolumns{0.5}{0.5}{
\begin{center}
\movie[width=0.9\textwidth, autostart, loop]{\includegraphics[width=0.9\textwidth]{p2-sac-valve}}{images/p2-sac-valve.mp4}
\end{center}
Manipulation from visual input, agent sees lower-right (took 20 hours to learn)
}{
\begin{center}
\movie[width=\textwidth, autostart, loop]{\includegraphics[width=\textwidth]{p2-sac-minitaur}}{images/p2-sac-minitaur.mp4}
\end{center}
Robust control of a legged robot (took 2 hours to learn)
}

\vspace{1em}

\footnotemark[1]{Haarnoja et al, 2018: ``Soft Actor-Critic Algorithms and Applications"}

\end{frame}

\begin{frame}{Anymal}

Hwangbo et al. used real data to learn a better simulator, and then used lots of simulator data to train complex locomotion and recovery policies with TRPO for the Anymal robot:

\begin{center}
\movie[width=0.8\textwidth, autostart, loop]{\includegraphics[width=0.8\textwidth]{p2-anymal}}{images/p2-anymal.mp4}
\end{center}

\footnotemark[1]{Hwangbo et al, 2018: ``Learning agile and dynamic motor skills for legged robots"}
\end{frame}

\begin{frame}{Learning Dexterity}

OpenAI et al. trained policies with PPO to dextrously manipulate a complex hand robot, using sim2real by domain randomization:

\begin{center}
\movie[width=0.8\textwidth, autostart, loop]{\includegraphics[width=0.8\textwidth]{p2-dexterity}}{images/p2-dexterity.mov}
\end{center}

\footnotemark[1]{OpenAI et al, 2018: ``Learning Dexterous In-Hand Manipulation"}

\end{frame}

\section{2.2: What are the challenges in modern RL?}

\begin{frame}{Some grand challenges for RL}

Let's talk about where RL fails, what we need that we aren't getting, and what research is happening.

\begin{itemize}
\item \textbf{Sample efficiency}: Modern deep RL algorithms are very inefficient with data, requiring in some cases 100x or more experience than a human to achieve good performance. How can we make them faster?
\pause
\item \textbf{Exploration}: Modern RL is terrible at environments where rewards are very rare---how can we get agents to try out new things?
\pause
\item \textbf{Generalization}: Policies trained for one task are brittle and cannot be used outside of training environment: how can we make them generalize?
\pause
\item \textbf{Safety}: How can we make sure that agents trained by deep RL behave in ways consistent with human preferences? {\color{red} (I'll let Dario cover this in his talk, but it's a critical challenge and I would be remiss not to mention it.)}
\end{itemize}

\end{frame}

\section{2.2.1: Research on Improving Sample Efficiency}

\begin{frame}{Combining Model-Free Improvements}
\twocolumns{0.5}{0.5}{
\textbf{Rainbow DQN}: combines...
\begin{itemize}
\item Dueling architecture
\item Double Q-Learning
\item Prioritized experience replay
\item N-step Q-Learning
\item Distributional Q-Learning
\item Parameter-space noise
\end{itemize}
}{
\includegraphics[width=\textwidth]{p2-rainbow}
}
\footnotemark[1]{Hessel et al, 2017: ``Rainbow: Combining Improvements in Deep Reinforcement Learning"}

\end{frame}

\begin{frame}{Accelerating Feature Learning with Unsupervised Learning}

\textbf{UNREAL} uses various unsupervised auxilliary tasks to speed up learning:

\twocolumns{0.5}{0.5}{
\begin{center}
\includegraphics[width=0.9\textwidth]{p2-unreal2}
\end{center}
}{
\begin{center}
\includegraphics[width=0.9\textwidth]{p2-unreal-labyrinth}
\end{center}
}
\vspace{1em}

\footnotemark[1]{Jaderberg et al, 2016: ``Reinforcement Learning with Unsupervised Auxiliary Tasks"}


%
%\begin{center}
%\movie[width=0.8\textwidth, autostart, loop]{\includegraphics[width=0.8\textwidth]{p2-unreal}}{images/p2-unreal.mp4}
%\end{center}

\end{frame}

\begin{frame}{Combining with Memory Mechanisms}

\textbf{MERLIN} combines unsupervised learning and attention-based memory to improve over baseline architectures:

\twocolumns{0.5}{0.5}{
\begin{center}
\includegraphics[width=0.9\textwidth]{p2-merlin1}
\end{center}
}{
\begin{center}
\includegraphics[width=0.9\textwidth]{p2-merlin2}
\end{center}
}

Note: this avenue may not help on arbitrary problems, but seems likely to help on sophisticated partially-observed ones
\vspace{1em}

\footnotemark[1]{Wayne et al, 2018: ``Unsupervised Predictive Memory in a Goal-Directed Agent"}

\end{frame}

\begin{frame}{Model-Based RL}

Techniques like \textbf{World Models}, that allow an agent to learn from simulated experience, are another way to bring down the needed number of real samples:

\begin{center}
\movie[width=0.8\textwidth, autostart, loop]{\includegraphics[width=0.8\textwidth]{world_models}}{images/world_models.mp4}
\end{center}

\footnotemark[1]{Ha and Schmidhuber, 2018: ``World Models"}

\end{frame}


\section{2.2.2: Research on Improving Exploration}

\begin{frame}{AKA, the Quest to Solve Montezuma's Revenge}

\begin{center}
\includegraphics[width=0.9\textwidth]{p2-montezuma}
\end{center}

\begin{itemize}
\item Famously hard Atari game
\item Most RL algorithms do terribly
\item Very sparse rewards, many actions lead to death
\item But humans can do just fine!
\end{itemize}

\end{frame}

\begin{frame}{Intrinsic Motivation}

Various \textbf{intrinsic motivation} schemes exist, which work by adding a non-environment-dependent reward:
%
\begin{equation*}
R(s,a,s') \to R(s,a,s') + \eta R_{int}(s,a,s')
\end{equation*}

Usually, intrinsic reward prefers \textbf{new experiences}. Examples:
%
\begin{itemize}
\item Pseudocount-based methods\footnote{Bellemare et al, 2016: ``Unifying Count-Based Exploration and Intrinsic Motivation"}: approximates number of state visitations with density model, rewards are:
%
\begin{equation*}
R_{int}(s,a,s') \propto \frac{1}{\sqrt{N(s)}}
\end{equation*}
\item Information gain\footnote{Houthooft et al, 2016: ``Variational Information Maximizing Exploration"}: approximates change in info state about environment model based on new experience
%
\begin{equation*}
R_{int}(s,a,s') \approx D_{KL}(P(\calE|\calD_t,s_{t+1})||P(\calE|\calD_t))
\end{equation*}
\end{itemize}

\end{frame}

\begin{frame}{Random Network Distillation}

\textbf{Random Network Distillation}\footnote{Burda et al, 2018: ``Exploration by Random Network Distillation"}: A new and appealingly simple form of intrinsic motivation:
%
\begin{equation*}
R_{int}(s) = \|f_{\phi}(s) - f_{\xi}(s)\|,
\end{equation*}
%
where $f_{\phi}$ is a random neural network (the target) and $f_{\xi}$ is trained to match $f_{\phi}$

\begin{itemize}
\item Exploits generalization properties of networks: $f_{\phi}$ matches $f_{\xi}$ on old states but not new states!
\end{itemize}


\begin{center}
\includegraphics[width=0.8\textwidth]{p2-rnd-atari}
\end{center}
\end{frame}

\begin{frame}{Go-Explore}

Ecoffet et al, 2018 identify a problem with intrinsic motivation that they call \textbf{detachment}:

\begin{center}
\includegraphics[width=0.8\textwidth]{p2-detach}
\end{center}

\end{frame}

\begin{frame}{Go-Explore}

They rectify detachment with \textbf{Go-Explore}\footnote{Ecoffet et al, 2018: ``Go-Explore: a New Approach for Hard-Exploration Problems"} algorithm:

\begin{center}
\includegraphics[width=0.8\textwidth]{p2-go-explore}

\includegraphics[width=0.4\textwidth]{p2-go-explore2}
\end{center}

\end{frame}



\section{2.2.3: Research on Generalization}

\begin{frame}{Hindsight Experience Replay}

Simple idea behind \textbf{HER}\footnote{Andrychowicz et al, 2017: ``Hindsight Experience Replay"}: 
\begin{itemize}
\item If you are trying to learn to go to a goal, goals are hard to get to. 
\item Unless you pretend that the place you ended up was where you meant to go: then you get to goals all the time!
\end{itemize}

\twocolumns{0.5}{0.5}{
\begin{center}
\includegraphics[width=0.9\textwidth]{p2-her}
\end{center}
}{
\begin{center}
\includegraphics[width=0.9\textwidth]{p2-her2}
\end{center}
}
\end{frame}


\begin{frame}{POET}

\textbf{Paired Open-Ended Trailblazer (POET)}\footnote{Wang et al, 2019: ``Paired Open-Ended Trailblazer (POET): Endlessly Generating Increasingly Complex and Diverse Learning Environments and Their Solutions
"} is an algorithm for evolving populations of agent-environment pairs towards a mix of performance and diversity

\begin{center}
\includegraphics[width=0.6\textwidth]{p2-poet1}

\includegraphics[width=0.6\textwidth]{p2-poet2}
\end{center}

Key idea: unboundedly complex environments challenge agents to develop robust, general behaviors
\end{frame}

\begin{frame}{Using Self-Play Systems}

Competitive multi-agent games create natural curricula that inspire general behaviors (because any fault will get exploited by the opponent). In self-play, agents play games against old versions of themselves to improve:\footnote{Bansal et al, 2017: ``Emergent Complexity via Multi-Agent Competition"}

\begin{center}
\movie[width=0.5\textwidth, autostart, loop]{\includegraphics[width=0.5\textwidth]{p2-selfplay}}{images/p2-selfplay.mov}
\end{center}


\end{frame}

\begin{frame}{Generalize by Meta-Learning}

\textbf{Model Agnostic Meta-Learning (MAML)}\footnote{Finn et al, 2017: ``Model-Agnostic Meta-Learning for Fast Adaptation of Deep Networks"} trains a model with an objective describing its adaptability: the model at optimum should only need a few gradient steps to quickly learn a new task.

\end{frame}

\section{2.3: Spinning Up in Deep RL}

\begin{frame}{Getting started}

Whether you intend to be a practitioner or a researcher:
\begin{itemize}
\item Brush up on math! Be familiar with vectors, matrices, gradients, gradient descent, random variables, expectations, variance, and a few other things.
\item Brush up on deep learning! \textbf{As much as you can get.}
\item Become familiar with TF or PyTorch! \textbf{You need to tinker to learn.}
\item Stay fresh on RL basics! \textbf{Fundamentals go a long way.}
\end{itemize}

\end{frame}

\begin{frame}{For maximum understanding: learn by doing}

\begin{itemize}
\item Write your own implementations
\item Simplicity is critical
\item Start with the basics: VPG, DQN, PPO, DDPG
\item Focus on understanding: it's the only way you will catch bugs
\end{itemize}
\end{frame}

\begin{frame}[fragile]{How silent are silent RL bugs? Pop quiz:}

\begin{lstlisting}[language=python]
    for t in range(total_steps):
        act = get_action(obs, epsilon)
        next_obs, rew, done, _ = env.step(act)
        replay_buffer.store(obs, act, rew, next_obs, done)
        ep_ret += rew
        ep_len += 1

        if done:
            epoch_rets.append(ep_ret)
            epoch_lens.append(ep_len)
            obs, rew, done, ep_ret, ep_len = env.reset(), 0, False, 0, 0

        if t > steps_before_training:
            batch = replay_buffer.sample_batch(batch_size)
            feed_dict = {obs_ph: batch['obs1'],
                         obs_targ_ph: batch['obs2'],
                         act_ph: batch['acts'],
                         rew_ph: batch['rews'],
                         done_ph: batch['done']
                         }
            step_loss, cur_q, _ = sess.run([loss, q_vals, train_op], feed_dict=feed_dict)
            epoch_losses.append(step_loss)
            epoch_qs.append(cur_q)

        if t % target_update_freq == 0:
            sess.run(target_update_op)

        epsilon = 1 + (final_epsilon - 1)*min(1, t/finish_decay)

        if (t - steps_before_training) % steps_per_epoch == 0 and (t - steps_before_training)>0:
            # handle logging outputs
\end{lstlisting}

\end{frame}

\begin{frame}{For maximum understanding: read the literature critically}

\begin{itemize}
\item If you read a paper, scour it: read all the ablations and search for tricks!
\item Skip the hype: find the thing the author's embarassed to admit about why the method doesn't solve the whole field yet
\item But don't overfit: sometimes researchers load more tricks than necessary and their idea is better than they realize
\end{itemize}
\end{frame}

\begin{frame}{For maximum understanding: "Let all results count with you, but none too much"}

\begin{itemize}
\item Study existing implementations of algorithms, but don't overfit to them: they made trade-offs you don't have to
\item Iterate fast in simple environments, but don't get too confident if your implementation (or new research idea) works in the simplest thing: everything that isn't bugged beats Cartpole, and so do some bugs
\item But if it doesn't work, assume there's a bug
\item Measure everything. But if your measurements say everything is fine while nothing's working, get new things to measure
\item When things start to work, lean in, scale up!
\item Keep these habits for research!
\end{itemize}

\end{frame}

\begin{frame}{If you want to do research}

\twocolumns{0.5}{0.5}{
\begin{itemize}
\item Do a deep dive in the literature
\item Find something you enjoy and care about---``keep only those research ideas which spark joy"
\item Consider approach 1: improve something that exists
\item Consider approach 2: attack an unsolved benchmark
\item Consider approach 3: invent a new problem
\item But whatever you pick, really do a thorough lit review to make sure it's new!
\end{itemize}
}{
\begin{center}
\includegraphics[width=\textwidth]{virtuous_cycle2}
\end{center}
}
\end{frame}
\end{document}